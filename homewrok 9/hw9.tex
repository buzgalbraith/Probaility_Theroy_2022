\documentclass[12pt,twoside]{article}
\usepackage[dvipsnames]{xcolor}
\usepackage{tikz,graphicx,amsmath,amsfonts,amscd,amssymb,bm,cite,epsfig,epsf,url}
\usepackage[hang,flushmargin]{footmisc}
\usepackage[colorlinks=true,urlcolor=blue,citecolor=blue]{hyperref}
\usepackage{amsthm,multirow,wasysym,appendix}
\usepackage{array,subcaption} 
% \usepackage[small,bf]{caption}
\usepackage{bbm}
\usepackage{pgfplots}
\usetikzlibrary{spy}
\usepgfplotslibrary{external}
\usepgfplotslibrary{fillbetween}
\usetikzlibrary{arrows,automata}
\usepackage{thmtools}
\usepackage{blkarray} 
\usepackage{textcomp}
\usepackage[left=0.8in,right=1.0in,top=1.0in,bottom=1.0in]{geometry}

%% Probability operators and functions
%
% \def \P{\mathrm{P}}
\def \P{\mathrm{P}}
\def \E{\mathrm{E}}
\def \Var{\mathrm{Var}}
\let\var\Var
\def \Cov {\mathrm{Cov}} \let\cov\Cov
\def \MSE {\mathrm{MSE}} \let\mse\MSE
\def \sgn {\mathrm{sgn}}
\def \R {\mathbb{R}}
\def \C {\mathbb{C}}
\def \N {\mathbb{N}}
\def \Z {\mathbb{Z}}
\def \cV {\mathcal{V}}
\def \cS {\mathcal{S}}

\newcommand{\RR}{\ensuremath{\mathbb{R}}}

\DeclareMathOperator*{\argmin}{arg\,min}
\DeclareMathOperator*{\argmax}{arg\,max}
\newcommand{\red}[1]{\textcolor{red}{#1}}
\newcommand{\blue}[1]{\textcolor{blue}{#1}}
\newcommand{\green}[1]{\textcolor{ForestGreen}{ #1}}
\newcommand{\fuchsia}[1]{\textcolor{RoyalPurple}{ #1}}

\newcommand{\wrnd}[1]{\widetilde{ #1 } }
\newcommand{\po}{\wrnd{\op{po}}  }

%
%% Probability distributions
%
%\def \Bern    {\mathrm{Bern}}
%\def \Binom   {\mathrm{Binom}}
%\def \Exp     {\mathrm{Exp}}
%\def \Geom    {\mathrm{Geom}}
% \def \Norm    {\mathcal{N}}
%\def \Poisson {\mathrm{Poisson}}
%\def \Unif    {\mathrm {U}}
%
\DeclareMathOperator{\Norm}{\mathcal{N}}

\newcommand{\bdb}[1]{\textcolor{red}{#1}}

\newcommand{\ml}[1]{\mathcal{ #1 } }
\newcommand{\wh}[1]{\widehat{ #1 } }
\newcommand{\wt}[1]{\widetilde{ #1 } }
\newcommand{\conj}[1]{\overline{ #1 } }
\newcommand{\rnd}[1]{\tilde{ #1 } }
\newcommand{\rv}[1]{ \rnd{ #1}  }
\newcommand{\rM}{\rnd{ m}  }
\newcommand{\rx}{\rnd{ x}  }
\newcommand{\ry}{\rnd{ y}  }
\newcommand{\rz}{\rnd{ z}  }
\newcommand{\ra}{\rnd{ a}  }
\newcommand{\rb}{\rnd{ b}  }
\newcommand{\rt}{\rnd{ t}  }
\newcommand{\rs}{\rnd{ s}  }


\newcommand{\rpc}{\widetilde{ pc}  }
\newcommand{\rndvec}[1]{\vec{\rnd{#1}}}

\def \cnd {\, | \,}
\def \Id { I }
\def \J {\mathbf{1}\mathbf{1}^T}

\newcommand{\op}[1]{\operatorname{#1}}
\newcommand{\setdef}[2]{ := \keys{ #1 \; | \; #2 } }
\newcommand{\set}[2]{ \keys{ #1 \; | \; #2 } }
\newcommand{\sign}[1]{\op{sign}\left( #1 \right) }
\newcommand{\trace}[1]{\op{tr}\left( #1 \right) }
\newcommand{\tr}[1]{\op{tr}\left( #1 \right) }
\newcommand{\inv}[1]{\left( #1 \right)^{-1} }
\newcommand{\abs}[1]{\left| #1 \right|}
\newcommand{\sabs}[1]{| #1 |}
\newcommand{\keys}[1]{\left\{ #1 \right\}}
\newcommand{\sqbr}[1]{\left[ #1 \right]}
\newcommand{\sbrac}[1]{ ( #1 ) }
\newcommand{\brac}[1]{\left( #1 \right) }
\newcommand{\bbrac}[1]{\big( #1 \big) }
\newcommand{\Bbrac}[1]{\Big( #1 \Big)}
\newcommand{\BBbrac}[1]{\BIG( #1 \Big)}
\newcommand{\MAT}[1]{\begin{bmatrix} #1 \end{bmatrix}}
\newcommand{\sMAT}[1]{\left(\begin{smallmatrix} #1 \end{smallmatrix}\right)}
\newcommand{\sMATn}[1]{\begin{smallmatrix} #1 \end{smallmatrix}}
\newcommand{\PROD}[2]{\left \langle #1, #2\right \rangle}
\newcommand{\PRODs}[2]{\langle #1, #2 \rangle}
\newcommand{\der}[2]{\frac{\text{d}#2}{\text{d}#1}}
\newcommand{\pder}[2]{\frac{\partial#2}{\partial#1}}
\newcommand{\derTwo}[2]{\frac{\text{d}^2#2}{\text{d}#1^2}}
\newcommand{\ceil}[1]{\lceil #1 \rceil}
\newcommand{\Imag}[1]{\op{Im}\brac{ #1 }}
\newcommand{\Real}[1]{\op{Re}\brac{ #1 }}
\newcommand{\norm}[1]{\left|\left| #1 \right|\right| }
\newcommand{\norms}[1]{ \| #1 \|  }
\newcommand{\normProd}[1]{\left|\left| #1 \right|\right| _{\PROD{\cdot}{\cdot}} }
\newcommand{\normTwo}[1]{\left|\left| #1 \right|\right| _{2} }
\newcommand{\normTwos}[1]{ \| #1  \| _{2} }
\newcommand{\normZero}[1]{\left|\left| #1 \right|\right| _{0} }
\newcommand{\normTV}[1]{\left|\left| #1 \right|\right|  _{ \op{TV}  } }% _{\op{c} \ell_1} }
\newcommand{\normOne}[1]{\left|\left| #1 \right|\right| _{1} }
\newcommand{\normOnes}[1]{\| #1 \| _{1} }
\newcommand{\normOneTwo}[1]{\left|\left| #1 \right|\right| _{1,2} }
\newcommand{\normF}[1]{\left|\left| #1 \right|\right| _{\op{F}} }
\newcommand{\normLTwo}[1]{\left|\left| #1 \right|\right| _{\ml{L}_2} }
\newcommand{\normNuc}[1]{\left|\left| #1 \right|\right| _{\ast} }
\newcommand{\normOp}[1]{\left|\left| #1 \right|\right|  }
\newcommand{\normInf}[1]{\left|\left| #1 \right|\right| _{\infty}  }
\newcommand{\proj}[1]{\mathcal{P}_{#1} \, }
\newcommand{\diff}[1]{ \, \text{d}#1 }
\newcommand{\vc}[1]{\boldsymbol{\vec{#1}}}
\newcommand{\rc}[1]{\boldsymbol{#1}}
\newcommand{\vx}{\vec{x}}
\newcommand{\vy}{\vec{y}}
\newcommand{\vz}{\vec{z}}
\newcommand{\vu}{\vec{u}}
\newcommand{\vv}{\vec{v}}
\newcommand{\vb}{\vec{\beta}}
\newcommand{\va}{\vec{\alpha}}
\newcommand{\vaa}{\vec{a}}
\newcommand{\vbb}{\vec{b}}
\newcommand{\vg}{\vec{g}}
\newcommand{\vw}{\vec{w}}
\newcommand{\vh}{\vec{h}}
\newcommand{\vbeta}{\vec{\beta}}
\newcommand{\valpha}{\vec{\alpha}}
\newcommand{\vgamma}{\vec{\gamma}}
\newcommand{\veta}{\vec{\eta}}
\newcommand{\vnu}{\vec{\nu}}
\newcommand{\rw}{\rnd{w}}
\newcommand{\rvnu}{\vc{\nu}}
\newcommand{\rvv}{\rndvec{v}}
\newcommand{\rvw}{\rndvec{w}}
\newcommand{\rvx}{\rndvec{x}}
\newcommand{\rvy}{\rndvec{y}}
\newcommand{\rvz}{\rndvec{z}}
\newcommand{\rvX}{\rndvec{X}}


\newtheorem{theorem}{Theorem}[section]
% \declaretheorem[style=plain,qed=$\square$]{theorem}
\newtheorem{corollary}[theorem]{Corollary}
\newtheorem{definition}[theorem]{Definition}
\newtheorem{lemma}[theorem]{Lemma}
\newtheorem{remark}[theorem]{Remark}
\newtheorem{algorithm}[theorem]{Algorithm}

% \theoremstyle{definition}
%\newtheorem{example}[proof]{Example}
\declaretheorem[style=definition,qed=$\triangle$,sibling=definition]{example}
\declaretheorem[style=definition,qed=$\bigcirc$,sibling=definition]{application}

%
%% Typographic tweaks and miscellaneous
%\newcommand{\sfrac}[2]{\mbox{\small$\displaystyle\frac{#1}{#2}$}}
%\newcommand{\suchthat}{\kern0.1em{:}\kern0.3em}
%\newcommand{\qqquad}{\kern3em}
%\newcommand{\cond}{\,|\,}
%\def\Matlab{\textsc{Matlab}}
%\newcommand{\displayskip}[1]{\abovedisplayskip #1\belowdisplayskip #1}
%\newcommand{\term}[1]{\emph{#1}}
%\renewcommand{\implies}{\;\Rightarrow\;}



\newcommand{\ru}{\rnd{ u}  }
\newcommand{\rd}{\rnd{ d}  }
%\newcommand{\rs}{\rnd{ s}  }
\newcommand{\ri}{\rnd{ i}  }
\newcommand{\re}{\rnd{ e}  }
\newcommand{\rQ}{\rnd{ q}  }
\newcommand{\rC}{\rnd{ c}  }


\begin{document}

\begin{center}
{\large{\textbf{Homework 9}} } \\%\vspace{0.2cm}\\
Due Nov 20 at 11 pm
\\
\end{center}
Unless stated otherwise, justify any answers you give.
You can work in groups, but each
student must write their own solution based on their own
understanding of the problem.

When uploading your homework to Gradescope you will have to
select the relevant pages for each question.  Please submit each
problem on a separate page (i.e., 1a and~1b can be on the same page but 1
and 2 must be on different pages).  We understand that this may be
cumbersome but this is the best way for the grading team to grade your
homework assignments and provide feedback in a timely manner.  Failure
to adhere to these guidelines may result in a loss of points.
Note that it may take some time to
select the pages for your submission.  Please plan accordingly.  We
suggest uploading your assignment at least 30 minutes before the deadline
so you will have ample time to select the correct pages for your
submission.  If you are using \LaTeX, consider using the minted or
listings packages for typesetting code.  
\\

\begin{enumerate}

\item (Short questions) 
Justify all your answers mathematically.
\begin{enumerate}
\item For any random variable $\ra$, can $\E\sqbr{\ra^2}$ be smaller than $\E\sqbr{\ra}^2$?  
\begin{itemize}
    \item note that $E[\Tilde{a}^2]-E[\Tilde{a}]^2=var(\Tilde{a})$
    \item further by definition of variance we can express $E[\Tilde{a}^2]-E[\Tilde{a}]^2=\\var(\Tilde{a})=E[(\Tilde{a}-E[\Tilde{a}])^2]$ without loss of generality we can assume that $\tilde{a}$ is discrete \item thus we have  $E[\Tilde{a}]^2-E[\Tilde{a}]^2=\\var(\Tilde{a})=E[(\Tilde{a}-E[\Tilde{a}])^2]=\Sigma_{a\in A}(a-E[\Tilde{a}])^2P_{(\Tilde{a}-E[\Tilde{a}])^2}((a-E[\Tilde{a}])^2)$
    \item further we know that $(a-E[\Tilde{a}])^2\geq 0$ since the quantity is squared, and  $P_{(\Tilde{a}-E[\Tilde{a}])^2}((a-E[\Tilde{a}])^2)\geq 0$ as it is a probability measure. 
    \item thus we can see $E[\Tilde{a}^2]-E[\Tilde{a}]^2=var(\Tilde{a})\geq 0$ which implies that $E[\Tilde{a}^2]\geq E[\Tilde{a}]^2$ for all random variables $\tilde{a}$
    
\end{itemize}


\item If $\ra$ and $\rb$ have the same distribution and are independent, is it true that $\E [\ra \rb] = \E\sqbr{\ra}^2$?

\begin{itemize}

\item $\forall \tilde{a}, \tilde{b}$ such that $\tilde{a}, \tilde{b}$ are indpendint and $P_{\tilde{a}}(a)=P_{\tilde{b}}(b)$ then we know that $E[\tilde{a}]=E[\tilde{b}]$
    \item we can express $E[\tilde{a}, \tilde{b}]=\Sigma_{a\in A}\Sigma_{b\in B}abP_{\tilde{a}, \tilde{b}}(a,b)=\Sigma_{a\in A}\Sigma_{b\in B}abP_{\tilde{a}}(a)P_{ \tilde{b}}(b)=\\\Sigma_{a\in A}P_{\tilde{a}}(a)\Sigma_{b\in B}aP_{ \tilde{b}}(b)=E[\tilde{a}]E[\tilde{b}]=E[\tilde{a}]E[\tilde{a}]=E[\tilde{a}]^2$
\end{itemize}
\item A teacher of a class of $n$ children asks their parents to leave a present under the Christmas tree in the classroom. The day after, each child picks a present at random. What is the expected number of children that end up getting the present bought by their own parents? (Hint: Define a random variable $\rnd{a}_i$ that is equal to one when kid $i$ gets the present bought by their own parents, and to zero otherwise.)

\begin{itemize}
    \item define $\tilde{p}$ as the number of remaining presents which is normally distributed between 0,n
    \item define $\tilde{a}$ as a binomial random variable representing the number of students who get the same present their parent left them.
    \item then let $\tilde{b}=b(\tilde{a},\tilde{p})$
    \item so we can write $E[\tilde{b}]=E[b(\tilde{a},\tilde{p})]$ further we know that a binomial can be writ en as a sum of Bernoulli rvs that is $\tilde{a}=\Sigma_{i=1}^{n}a_i$ \item so thus we have $E[\tilde{b}]=E[b(\tilde{a},\tilde{p})]=E[\Sigma_{i=1}^{n}\tilde{a_i}\tilde{p})]=\Sigma_{i=1}^{n}E[\tilde{a_i}\tilde{p})]=\Sigma_{i=1}^{n}\Sigma_{p=1}^{n}a_iP_{\tilde{a_i}\tilde{p}}(a_i,p)P_{\tilde{p}}(p)\\=\Sigma_{p=1}^{n}\Sigma_{i=1}^{n}a_i\frac{1}{p}\frac{1}{n}=\Sigma_{p=1}^{n}\frac{1}{p}\frac{1}{n}=n*\frac{1}{n}\Sigma_{p=1}^{n}\frac{1}{p}=\Sigma_{p=1}^{n}\frac{1}{p}=1$
    
\end{itemize}
\end{enumerate}

  \item (Mean of a function) Let $\ra$ be a discrete random variable with range $A$ and pmf $p_{\ra}$ and let $\rnd{b}:=h(\ra)$ be a random variable with range $B$ obtained by applying a deterministic function $h:\R\rightarrow \R$ to $\ra$. Prove that our definition of $\E(h(\ra))$ is sound, i.e. that
\begin{align}
\E(\rnd{b}) & := \sum_{b \in B} b \, p_{\rb}(b) \\
& = \sum_{a \in A} h(a) p_{\rnd{a}} (a).
\end{align}

\begin{itemize}
    \item first we want to show that b=h(a) $\forall b\in B$. this holds direclty from the defention. 
    \item next we want to show that for the values of a that map to some value of b $P_{\tilde{b}}(b)=P_{\tilde{a}}h(a)$
    \begin{itemize}
        \item note that we can expres $P_{\tilde{b}}(b)=P_{h(\tilde{a})}(h(a))$ for some subset of $a\in A$ as we understand h is deterministic, and P is a valid probability funciton 
        \item we can see that $P_{\tilde{b}}(b)=P_{h(\tilde{a})}(h(a))=P_{\tilde{a}}P(a)$
    \end{itemize}
    \item thus we can see that for any $b\in B$ there exists a $a\in A$ such that $aP_{\tilde{a}}(a)=bP_{\tilde{b}}(b)$
    \item furhter we know that $h:A\rightarrow B$ thus H will map all values in A to B, and there will always be some value in A which maps to all values in A 
    \item thus we can see that $\E(\rnd{b}) & := \sum_{b \in B} b \, p_{\rb}(b) \\
 = \sum_{a \in A} h(a) p_{\rnd{a}} (a).$ will hold
 
\end{itemize}


\item (Standard deviations) 
\begin{enumerate}
\item Compute the standard deviation of the total number of animals (cats + dogs) per household in Example 7.6.
\begin{itemize}
    \item from the definition of variance we can compute $var(\tilde{c}+\tilde{d})=E[(\tilde{c}+\tilde{d})^2]-E[\tilde{c}+\tilde{d}]^2$ 
    \item so first lets find $E[(\tilde{c}+\tilde{d})^2]=E[\tilde{c}^2+2\tilde{c}\tilde{d}+\tilde{d}^2]=E[\tilde{c}^2]+2E[\tilde{c}\tilde{d}]+E[\tilde{d}^2]$ 
    \item now we can find each of these one at a time form the data. 
    \item $E[\tilde{c}^2]=\Sigma_{c=0}^{3}c^2p_{\tilde{c}}(c)=1(.15+.05+.02)+4(.1+.03)+9(.05)=1.19$
    \item$2E[\tilde{c}\tilde{d}]=2\Sigma_{c=0}^{3}\Sigma_{d=0}^{2}cdp_{\tilde{c},\tilde{d}}(c,d)=2(1(.05)+2(.02)+2(.03))=.3$
    \item $E[\tilde{d}^2]=\Sigma_{c=0}^{2}d^2p_{\tilde{d}}(d)=1(.20+.05+.03)+4(.05+.02)=0.56$
    \item $E[c+d]^2=(1.05)^2=1.1025$
    \item $var(\tilde{c}+\tilde{d})=E[(\tilde{c}+\tilde{d})^2]-E[\tilde{c}+\tilde{d}]^2=1.19+.3+.56-1.1025=0.9475$
    \item then we can finally get our standard deviation as $std(\tilde{c}+\tilde{d})=\sqrt{var(\tilde{c}+\tilde{d})}=\sqrt{.9457}=0.97339$
    
\end{itemize}


\item Each time a basketball player attempts a shot, it is a 2-point shot with probability $\alpha$ and a 3-point shot with probability $1-\alpha$ for $0 \leq \alpha \leq 1$. The probability that she makes the 2-point shot is 0.5. The probability that she makes the 3-point shot is 0.4. Compute the standard deviation of the points she scores per shot, as a function of $\alpha$. What are the maximum and minimum values of the standard deviation, and at what values of $\alpha$ are they attained?   

\begin{itemize}
    \item let $\tilde{a}$ take value 1 if the player shoots a two pointer 
    \item let $\tilde{s}$ take value 1 if the shot goes in 
    \item we can see that $P_{\tilde{a}}(a=1)=\alpha$
    \item and $P_{\tilde{a}}(a=0)=1-\alpha$
    and $p_{\tilde{s}|\tilde{a}}(s=1|a=1)=.5$ and$p_{\tilde{s}|\tilde{a}}(s=1|a=0)=.4$ 
    \item let $\tilde{p}=p(\tilde{s},\tilde{a})$ represent the number of points a player gets
    \item $var(\tilde{p})=E[p^2]-E[p]^2$
    \item we can compute $E[p^2]=\Sigma_{a=0}^{1}\Sigma_{s=0}^{1}p(s,a)^2p_{\tilde{s},\tilde{a}}(s,a)=2^2(\alpha)(.5)+3^2(1-\alpha)(.4)=3.6-1.6\alpha$
    \item then we can compute$E[p]^2=\Sigma_{a=0}^{1}\Sigma_{s=0}^{1}p(s,a)2p_{\tilde{s},\tilde{a}}(s,a)^2=(2(\alpha)(.5)+3(1-\alpha)(.4))^2=(1.2-.2\alpha)^2=(1.44+4.24\alpha+.4\alpha^2$
    \item so we finally get $var(\tilde{p})=2.16+2.64\alpha-.4\alpha^2$
    \item then we can compute the standard devation as $\sqrt{var(\tilde{p})}=\sqrt{2.16-2.12\alpha-.4\alpha^2}$
    \item this is an inverted parabola. 
    \item we find that we get the highest standard devation at $\alpha=0$ which yields a std=1.47
    \item and the lowest standard devation we observe is at $\alpha=1$ with a standard devation of .08
\end{itemize}

\end{enumerate}

\item (Air Quality) AQI is an index that indicates the level of air pollution (the higher AQI, the more pollution in the air). The file \textit{aqi.npy} records the AQI of a city over a period of time. Compute the mean and standard deviation of AQI, and also a boxplot of the data. Are the mean and standard deviation of AQI a good measure of the level of air pollution in a city?

\end{enumerate}
\end{document}
